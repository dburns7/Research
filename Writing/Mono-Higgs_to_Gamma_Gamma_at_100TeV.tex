\documentclass[twoside]{article}

\usepackage{lipsum}
\linespread{1.05} % Line spacing
\usepackage{microtype} % Slightly tweak font spacing
\usepackage[hmarginratio=1:1,top=25.4mm,right=25.4mm,columnsep=20pt]{geometry} % Document margins
\usepackage{multicol} % Used for the two-column layout
\usepackage[hang, small,labelfont=bf,up,textfont=it,up]{caption} % Custom captions under/above floats in tables or figures
\usepackage{booktabs} % Horizontal rules in tables
\usepackage{float} % Required for tables and figures in the multi-column environment - they need to be placed in specific locations with the [H] (e.g. \begin{table}[H])
\usepackage{hyperref} % For hyperlinks in the PDF
\usepackage{paralist} % Used for the compactitem environment which makes bullet points with less space between them
\usepackage{abstract} % Allows abstract customization
\renewcommand{\abstractnamefont}{\normalfont\bfseries} % Set the "Abstract" text to bold
\renewcommand{\abstracttextfont}{\normalfont\small\itshape} % Set the abstract itself to small italic text
\usepackage{titlesec} % Allows customization of titles
\renewcommand\thesection{\Roman{section}} % Roman numerals for the sections
\renewcommand\thesubsection{\Roman{subsection}} % Roman numerals for subsections
\titleformat{\section}[block]{\large\scshape\centering}{\thesection.}{1em}{} % Change the look of the section titles
\titleformat{\subsection}[block]{\large}{\thesubsection.}{1em}{} % Change the look of the section titles
\usepackage{graphicx}
\graphicspath{ {/home/dustin/Research/MonoHiggs_AA_100TeV/} }

\title{\vspace{-15mm}\fontsize{16pt}{10pt}\selectfont\textbf{Mono-Higgs Dark Matter Phenomenology at 100 TeV}} % Article title
\author{
\large
\textsc{Dustin Burns}\thanks{University of California, Davis}\\[2mm]
%\normalsize University of California, Davis \\ 
%\normalsize \href{mailto:dburns@ucdavis.edu}{dburns@ucdavis.edu} 
\vspace{-5mm}
}

\begin{document}

\maketitle 

\begin{abstract}

\noindent The phenomenology of the Higgs plus missing transverse energy -- mono-Higgs -- detector signiture is studied at a center of mass energy of 100 TeV, the operating center of mass energy of a proposed new hadron collider. The detection sensitivity in the Higgs to two photon channel is shown to be increased over that found in previous studies at center of mass energies of 8 TeV and 14 TeV,  further motivating the design and construction of a 100 TeV collider.  
%\noindent \lipsum[1] % Dummy abstract text

\end{abstract}

\begin{multicols}{2} % Two-column layout throughout the main article text

\section{Introduction}

%\lettrine[nindent=0em,lines=3]{L} orem ipsum dolor sit amet, consectetur adipiscing elit.
%\lipsum[2-3] % Dummy text
\noindent The dark matter (DM) mono-Higgs signature consists of a pair of dark matter particles recoiling against a Higgs boson (H), which can decay through the standard processes. The decay products of H are reconstructed in the detector and the DM escapes undetected, resulting in missing transverse energy (MET).\\
\indent The theoretical sensitivity to the mono-H signiture at the LHC at center of mass energies of 8 TeV and 14 TeV is 100 fb -- 1 pb over a range of dark matter (DM) models [1]. These models consist of effective field theories (EFTs) and simplified models, which include a new massive mediator particle, plus the standard model (SM). The H diphoton decay channel was found to be the most sensitive at the 8 TeV and 14 TeV LHC.\\
\indent In contrast to other mono-X processes, which yield a particle X (photon [2], jet [3], lepton [4], etc) via initial state radiation (ISR) of X, the mono-H process couples DM to the Higgs at the same vertex. Therefore, this process explores the direct coupling of DM to the SM, a very exciting prospect theoretically. \\
\indent While LHC mono-H analyses at 8 TeV are currently underway at both CMS [5] and ATLAS [6], and 13 TeV analyses are being prepared for multiple Higgs decay channels, the current study will explore the sensitivity of these DM models at a proposed new collider operating at a center of mass (COM) energy of 100 TeV. Since the previous study has shown the Higgs diphoton decay channel to be the most sensitive, this channel will be the focus of the current study. \\

\section{Methods}

To explore the phenomenology of the DM mono-H to diphoton signature at 100 TeV, the analysis will parallel the 8 TeV and 14 TeV analyses given in section IIIA of [1]. Background and signal Monte Carlo (MC) samples are produced using the matrix element generator MADGRAPH5 [7], using PYTHIA [8] for showering and DELPHES [9] for the detector simulation. These samples are analyzed using DELPHES plugins to the ROOT [10] software package. \\
\indent The backgrounds that are expected to be dominiant in the mono-H to diphoton channel are (1) ZH, Z $\rightarrow \nu \bar{\nu}$, (2) WH, W $\rightarrow l \nu$, (3) H $\rightarrow \gamma \gamma$, (4) Non-resonant $\gamma \gamma$ production, and (5) Z$\gamma\gamma$, Z $\rightarrow \nu \bar{\nu}$. MET arrises from neutrinos escaping the detector and from the mismeasurement of leptons and photons. \\
\indent The signal models used to generate MC are the benchmark models given in Table 1 of [1]. These include EFT models with operators up to dimension eight, and simplified models containing either a Z' boson or a new scalar S which couples only to the H field. The mass of the DM particle $\chi$ is varied over the values $m_{\chi} = 1, 10, 100, 500, 1000$ GeV. \\
\indent The cross sections and branching ratios for the background channels (1), (2), and (5) at 100 TeV are obtained from [11] and [12]. The cross sections for background channels (3) and (4) are obtained using MADGRAPH5, and corrected to NLO with k-factors obtained from [13] and [14]. The cross section times branching ratio used for a signal model will be shown in plot legends or table description where necessary. Kinematic distributions are normalized assuming 3000 $fb^{-1}$ of data. \\
\indent Expected limits are set using the HiggsAnalysis CombinedLimit tool [15] using a non-shape based
Bayesian $95\%$ one-sided credible interval (upper limit) as a function of $m_{\chi}$. \\

% Bullet list: 
%\begin{compactitem}
%\item Donec dolor arcu, rutrum id molestie in, viverra %sed diam
%\end{compactitem}
%\lipsum[4] % Dummy text

\section{Results}

The diphoton invariant mass and MET kinematic variables used in the event selection cuts are shown in Figures 1 and 2, respectively. The selected regions are bracketed by vertical lines. The additional cuts made in the event selection are: \\
 
\begin{compactitem}
\item Exactly two final state photons, each with $p_{T} > 20$ and $|\eta| < 2.5$
\item $m_{\gamma\gamma} \in [116, 136]$ GeV
\item Final state leptons have $p_{T} < 20$ and $|\eta| > 2.5$
\item MET $>$ 100 GeV
\end{compactitem}

\begin{figure}[H]
\centering
\includegraphics[width=0.5\textwidth]{mgg}
\caption{Dilepton invariant mass distributions, normalized to one.}
\end{figure}

\begin{figure}[H]
\centering
\includegraphics[width=0.5\textwidth]{met_unitnorm}
\caption{Missing transverse energy distributions, normalized to one.}
\end{figure}

The event yields for the background channels and a representative benchmark model are given in Table 1. The significance for this scenerio is $S / \sqrt{B} = 1$ \\

\begin{table}[H]
\centering
\begin{tabular}{llr}
\toprule
%\multicolumn{2}{c}{Name} \\
%\cmidrule(r){1-2}
Channel & Yield \\
\midrule
ZH, Z $\rightarrow \nu \bar{\nu}$ & 1 \\
WH, W $\rightarrow l \nu$ & 1 \\
H $\rightarrow \gamma \gamma$ & 1 \\
$\gamma \gamma$ & 1 \\
Z$\gamma\gamma$, Z $\rightarrow \nu \bar{\nu}$ & 1 \\
\midrule
Total Background & 1\\
Total Signal & 1\\
\bottomrule
\end{tabular}
\caption{Event Yields}
\end{table}

\indent The MET distributions before and after selection cuts are shown in Figures 3 and 4, respectively. \\

\begin{figure}[H]
\centering
\includegraphics[width=0.5\textwidth]{met_precuts}
\caption{Missing transverse energy before selection cuts, normalized to 300 $fb^{-1}$}
\end{figure}

\begin{figure}[H]
\centering
\includegraphics[width=0.5\textwidth]{met_postcuts}
\caption{Missing transverse energy after selection cuts, normalized to 300 $fb^{-1}$}
\end{figure}

The selection efficiency and cross section upper limit for a set of benchmark models versus $m_{\chi}$ is shown in Figures 5 and 6, respectively. \\

% Equation
%\begin{equation}
%\label{eq:emc}
%e = mc^2
%\end{equation}

\section{Discussion}

The event selection criteria reduce the background events by ?? orders of magnitude while only reducing the signal events by a factor of ?? as shown by the selection efficiencies in Figure 5. This reiterates the point that the diphoton channel is very clean despite the H to diphoton cross section being small compared to other decay channels. \\
\indent From Figure 6, the sensitivity of this process ranges from ?? to ?? for the various models and values of $m_{\chi}$, a factor of ?? better than the expected sensitivity at the 14 TeV LHC. \\


\section{Citations}

%\begin{thebibliography}{99} % Bibliography - this is intentionally simple in this template

[1] Carpenter, L., DiFranzo, A., Mulhearn, M., Shimmin, C., Tulin, S., Whiteson, D. (2014). Mono-Higgs-boson: A new collider probe of dark matter. Phys. Rev. D, 89(7), 75017. doi:10.1103/PhysRevD.89.075017 \\ \relax
[2] CMS Collaboration. (2014). Search for new phenomena in monophoton final states in proton-proton collisions at sqrt(s) = 8 TeV. High Energy Physics - Experiment. Retrieved from http://arxiv.org/abs/1410.8812 \\ \relax
[3] CMS Collaboration. (2014). Search for dark matter, extra dimensions, and unparticles in monojet events in proton-proton collisions at sqrt(s) = 8 TeV. High Energy Physics - Experiment. Retrieved from http://arxiv.org/abs/1408.3583 \\ \relax
[4] CMS Collaboration. (2014). Search for physics beyond the standard model in final states with a lepton and missing transverse energy in proton-proton collisions at sqrt(s) = 8 TeV. High Energy Physics - Experiment. Retrieved from http://arxiv.org/abs/1408.2745 \\ \relax
[5] \\ \relax
[6] \\ \relax 
[7] J. Alwall, M. Herquet, F. Maltoni, O. Mattelaer, and
T. Stelzer, JHEP 1106, 128 (2011), 1106.0522. \\ \relax
[8] T. Sjostrand, S. Mrenna, and P. Z. Skands, JHEP 0605,
026 (2006), hep-ph/0603175.\\ \relax
[9] S. Ovyn, X. Rouby, and V. Lemaitre (2009), 0903.2225\\ \relax
[10] \\ \relax
[11] \url{https://twiki.cern.ch/twiki/bin/view/LHCPhysics/CERNYellowRepor    tPageBR2#Higgs_2_gauge_bosons} \\ \relax
[12] \url{https://twiki.cern.ch/twiki/bin/view/LHCPhysics/Hi    ggsEuropeanStrategy#SM_Higgs_decay_branching_ratio_M} \\ \relax
[13] G. Bozzi, F. Campanario, M. Rauch, and D. Zeppenfeld,
Phys.Rev. D84, 074028 (2011), 1107.3149. \\ \relax
[14] G. Aad et al. (ATLAS Collaboration), Phys.Rev. D85,
012003 (2012), 1107.0581. \\ \relax
[15]
%\bibitem[Figueredo and Wolf, 2009]{Figueredo:2009dg}
%Figueredo, A.~J. and Wolf, P. S.~A. (2009).
%\newblock Assortative pairing and life history strategy - %a cross-cultural
%  study.
%\newblock {\em Human Nature}, 20:317--330.
 
%\end{thebibliography}

\end{multicols}
\end{document}
